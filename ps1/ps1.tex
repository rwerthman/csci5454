\documentclass[12pt]{article}
\begin{document}
\title{CSCI 5454: PS1}
\author{Robert Werthman}
\date{}
\maketitle

\section*{1.}

\subsection*{}
Let's say these algorithms solve an array sorting problem.\\
\begin{itemize}
\item Let algorithm $A$ be bubblesort with a worst-case runtime of $n^2$.\\
\item Let algorithm $B$ be mergesort with a worst-case runtime of $n*log(n)$.\\
\item Let $C$ be the newly designed sorting algorithm with a worst-case runtime of $h(n)$.\\
\end{itemize}
In this case, $O(min(f(n),g(n)))$ will become $O(n*log(n))$ because it is the smaller of the two runtimes.\\
If $h(n)$ is $log(n)$ then $h(n)$ achieves the running time $O(min(f(n),g(n)))$ because $log(n)$ does not grow faster than $n*log(n)$ and is therefore bounded above by it.\\

\subsection*{}
Yes, you can achieve a running time exactly $min(f(n),g(n))$. Algorithm $C$ would need to be designed in such a way that its running was equal to $min(f(n),g(n))$.\\

\section*{2.}

\subsection*{}
\textbf{Proposition/Claim: } For any real constants $a$ and $b$, where $b > 0$, the asymptotic relation $(n+a)^b = \Theta(n^b)$ is true.\\
\\
\textbf{Theorem: }The asymptotic relation $(n+a)^b = \Theta(n^b)$ is true iff:
\begin{itemize}
\item There exists positive constants $c_1, c_2, n_0$ s
uch that $0 \le c_1(n^b) \le (n+a)^b \le c_2(n^b)$ for all $n \ge n_0$.
\end{itemize}In order to prove the proposition above we must find some constants $c_1, c_2, n_0$ to satisfy the above bulleted sentence.\\
\\
\textbf{Proof: }\\
First we want to find the floor and ceiling of $n+a$ so we can create an inequality similar to the one in the theorem above.
\begin{enumerate}
\item If $|a| \le n$ then we can say that $n+a \le n+|a| \le 2n$ (Ceiling of $n+a$).
\item If $|a| \le \frac{1}{2}n$ then we can say that $n+a \ge n-|a| \ge \frac{1}{2}n$ (Floor of $n+a$). 
\end{enumerate}
Now if $2|a| \le n$ then we can combine the floor and ceilings into an compound inequality that holds true :
$$
0 \le \frac{1}{2}n \le n+a \le 2n
$$
The only thing missing from this new equation is a power of $b$.  Raising the new equation to a power of $b$ gives:
$$
0 \le (\frac{1}{2}n)^b \le (n+a)^b \le (2n)^b \Rightarrow 0 \le (\frac{1}{2})^bn^b \le (n+a)^b \le (2)^bn^b
$$  
Extracting the constants $c_1,c_2,n_0$ from this equation yields $c_1 = (\frac{1}{2})^b$, $c_2 = 2^b$, and $n_0 = 2|a|$ since $n \ge 2|a|$.  These represent one solution.
\section*{3.}
$f(n) = \Omega{g(n)}$ means that for all values to the right of some $n_0$ the value of $f(n)$ is on or above $cg(n)$.\\
\begin{center}
\begin{tabular}{|c|c|c|c|c|c|c|c|c|c|c|c|}
\hline
$n!$&$e^n$&$(\frac{3}{2})^n$&$(lg\,n)!$&$n^2$&$n\,lg\,n$&$lg(n!)$&n&$(\sqrt{2})^{lg\,n}$&$2^{lg*n}$&$n^{1/lg\,n}$&1\\
\hline
\end{tabular}
\end{center}
\subsection*{Equivalence Classes}
$lg(n!) = \Theta(n\,lg\,n)$\\
$n^{1/lg\,n} = \Theta(1)$
\end{document}  