\documentclass[12pt]{article}

\usepackage{graphicx,amsmath}

\title{CSCI 5454 Final Project: AVL Tree}
\author{Robert Werthman}
\date{}

\begin{document}

\maketitle

\newpage
\tableofcontents

\newpage
\addcontentsline{toc}{section}{Introduction}
\section*{Introduction}

\addcontentsline{toc}{subsection}{What is an AVL Tree?}
\subsection*{What is an AVL Tree?}

An AVL tree is a binary search tree that is ``self-balancing''.  This means
after each operation, like an insertion or deletion, on the tree the heights of 
each node's children differ by at most 1.  The height of a node is the number of
nodes in the longest path from it to a leaf node.  A leaf node
always has a height of 0. Its parent node, if the leaf
node was the parent's only child, would have a height of 1.  An AVL tree is
``self-balancing'' because after each operation on the tree the heights and balance of the nodes are readjusted by the tree,
itself \cite{wiki:avl}.

\addcontentsline{toc}{subsection}{What problems does it solve?}
\subsection*{What problems does it solve?}
AVL trees, like binary trees, are used for storing and retrieving
information.  Their advantage is that they can perform these operations faster
than if the information was stored in an array.  As will be shown later in this
paper, storing and retrieving takes $log(n)$ time for an AVL tree
while performing the same operations on an array could take up to $n$ time.

\addcontentsline{toc}{subsection}{Where is it used?}
\subsection*{Where is it used?}
Red-black trees, another kind of self-balancing binary search tree, are
typically used instead of AVL trees in real world applications
\cite{wiki:red-black}.
Red-black trees can be found in the C++ Standard Library (std) as the underlying data structure
for the std::map and std::set containers and it is reasonable to think that AVL
trees could be used instead.

\addcontentsline{toc}{section}{Mathemetical Analysis of Correctness, Runtime, and Space}
\section*{Mathemetical Analysis of Correctness, Runtime, and Space}

\addcontentsline{toc}{subsection}{Correctness}
\subsection*{Correctness}
The correctness of the AVL tree relies on maintaining two invariants
\begin{enumerate}
  \item The height of a node is the number of nodes in the longest path from it
  to a leaf node.
  \item The heights of each node's children differ by at most 1.  This is
  demonstrated by the equation $|x-y| \le 1$ where x is the height of the left
  child node and y is the height of the right child node.
\end{enumerate}
\begin{figure}[h]
\caption{An AVL tree with a single node}
\includegraphics[width=9cm]{single_node_tree.png}
\centering
\end{figure}
\noindent
To show these invariants are maintained we start with a single, leaf node in a
tree as the base case as shown in Figure 1.  First, we let the height of a non
existent node be equal to -1.
This means a single node in a tree has two non existent children nodes each
with a height of -1.  To find the height of the single node in the tree we take the max of the
heights of the children and then add 1 to it to always maintain the first
invariant.  In this case max(-1, -1) = -1 and adding 1 to this result gives a
height of 0 for the single node in the tree.  To make sure the second invariant
is maintained we use the given equation
$$
|x-y| \le 1
$$
In the case of a single, leaf node the height of the left child is
-1 and the height of the right child is -1.  Substituting these values in for
$x$ and $y$ produces the equation
\begin{align*}
|(-1)-(-1)| &= 0\\
&\le 1
\end{align*}
Since $0 \le 1$ the second invariant is maintained.\\
\\
Next it must be inductively shown that the two invariants hold for every AVL
tree with size greater than 1 node.  Concerning the first invariant, every time
a new node is inserted it will be leaf node with a height of 0.  

\addcontentsline{toc}{subsection}{Runtime}
\subsection*{Runtime}

\addcontentsline{toc}{subsection}{Space}
\subsection*{Space}

\addcontentsline{toc}{section}{Numerical Characterization of Runtime and Space}
\section*{Numerical Characterization of Runtime and Space}

\addcontentsline{toc}{subsection}{Description of the code invloved in the Numerical Characterization}
\subsection*{Description of the code invloved in the Numerical Characterization}
To show the numerical characterization of the space and time performance of an
AVL tree, I created a randomized input generator to show the runtime and space
of the three operations that can be performed on the AVL tree: insertion,
deletion, and search.  I ran 12 iterations for each of these tree operations,
varying the number of nodes $n$ in the trees by a factor of 2.  I ran each of
these iterations 50 times each to find the average number of operations given a
value of $n$.  $n$ takes on all of the values in the set
$\{2^4,\,2^5,\ldots,2^{16}\}$ for each tree operation.
The keys for each of the nodes in the $n$ node tree were randomly chosen from 
the set $\{0,\,,1,\ldots,2^n$ and then inserted into the tree to create a tree of size $n$.\\
\\
Once the tree was generated for an iteration, I then chose a random key
from the set of keys that were already in the tree, and I either searched for
it, insertd it, or deleted it.  To keep track of the runtime when performing
these operations on the tree, I kept a global variable as a counter and
incremented it every time an atomic operation occurred.  To keep track of the space used when
performing these operations on the tree, I kept a global variable as a counter
and incremented it when a node was inserted and decremented it when a node was
removed from the tree.  These global variables were reset after each
iteration.\\
\\
Once all the iterations were complete for a specific operation, I took the $n$
for each iteration and the values of the global variables for each iteration and
graphed them as a function $f(x)$.  The values of $n$ were placed on the
x-axis and the values of the globabl variables were placed on the y-axis.  The graphs use the logarithmic
scale instead of the linear scale because the logarthmic scale more clearly
shows what $n$ and the values of the global variables are doing.
I then graphed one other line representing the function $g(x)$, which is
the function that bounds $f(x)$. $g(x)$ is then
multiplied by a constant, $c_2$, producing a separate line.
\addcontentsline{toc}{subsection}{Characterization of Runtime}
\subsection*{Characterization of Runtime}

\addcontentsline{toc}{subsubsection}{Runtime of Search}
\subsubsection*{Runtime of Search}
\begin{figure}[h]
\caption{Graph of the runtime for searching for a key in different-sized AVL
trees}
\includegraphics[width=9cm]{search_runtime.png}
\centering
\end{figure}
\noindent
The above graph shows the runtime for searching for a key in different-sized
avl trees.
The function that bounds the runtime of search is $g(n)=log(n)$ multiplied by
the constant $c_2=2$.
As can be seen in the graph the
runtime of search is $O(log(n))$, with the function $log(n)$ bounding the
search runtime from above.  This means given an AVL tree of size $n$, it will
take $O(log(n))$ operations to search for key in the tree.

\addcontentsline{toc}{subsubsection}{Runtime of Insert}
\subsubsection*{Runtime of Insert}
\begin{figure}[h]
\caption{Graph of the runtime for inserting a key into different-sized AVL
trees}
\includegraphics[width=9cm]{insert_runtime.png}
\centering
\end{figure}
\noindent
The above graph shows the runtime for inserting a key into different-sized AVL
trees.  The function that bounds the runtime of insert is $g(n)=log(n)$
multiplied by the constant $c_2=4$.  As can be seen in the graph the runtime of
insert is $O(log(n))$, with the function $log(n)$ bounding the insert runtime
from above.  This means given an AVL tree of size $n$, it will take $O(log(n))$
operations to insert a key into the tree.

\addcontentsline{toc}{subsubsection}{Runtime of Delete}
\subsubsection*{Runtime of Delete}
\begin{figure}[h]
\caption{Graph of the runtime for deleting a key from different-sized AVL
trees}
\includegraphics[width=9cm]{delete_runtime.png}
\centering
\end{figure}
\noindent
The above graph shows the runtime of deleting a key from different-sized AVL
trees.  The function that bounds the runtime of delete is $g(n)=log(n)$
multiplied by the constant $c_2=4$.  As can be seen in the graph the runtime of
delete is $O(log(n))$, with the function $log(n)$ bounding the delete runtime
from above.  This means given an AVL tree of size $n$, it will take $O(log(n))$
operations to delete a key from the tree.

\addcontentsline{toc}{subsection}{Characterization of Space Usage}
\subsection*{Characterization of Space Usage}
\begin{figure}[h]
\caption{Graph of the space used by different-size AVL trees}
\includegraphics[width=9cm]{space_used.png}
\centering
\end{figure}
\noindent
The above graph shows the space used by different-size AVL trees.  The functions
that bounds the space used is $g(n) = n$ multiplied by the constant $c_2 = 2$. 
As can be seen in the graph the space that is used by an AVL tree is $O(n)$,
with the function $n$ bounding the space used from above.  This means that given
an AVL tree of size $n$, there will be exaclty $n$ nodes in the tree.  With a
tree of size $n$, if a node is inserted there will be $n+1$ nodes in the tree. 
With a tree of size $n$, if a node is deleted there will be $n-1$ nodes in the
tree.

\addcontentsline{toc}{section}{Extensions, Improvements, and Recent Work}
\section*{Extensions, Improvements, and Recent Work}

\newpage
\addcontentsline{toc}{section}{References}

\bibliographystyle{unsrt}
\bibliography{sample}


\end{document}